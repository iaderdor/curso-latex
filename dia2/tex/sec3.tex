\makeatletter
\def\input@path{{../}}
\makeatother

\documentclass[../preambulo.tex]{subfiles}


\begin{document}


\section{AMSLaTeX}

% Este es un ejemplo de cómo consturir matrices.
\[
	\begin{pmatrix}
		1 & 0 & 0 \\
		0 & 1 & 0 \\
		0 & 0 & 1 
	\end{pmatrix}
\]

% Este ambiente del modo matemático nos permite escribir varias ecuaciones en el mismo abiente que ocupen varias líneas. No están alineadas.
\begin{gather*}
	a^2 = b^2 + c^2 \\
	E = m c^2
\end{gather*}

% Este ambiente nos permite escribir ecuaciones en varias líneas alineadas. Siempre se alinean con respecto al carácter que vaya a continuación de &
\begin{align*}
	a^2 &= b^2 + c^2 \\
	E &= m c^2	
\end{align*}


% Aligned nos permite alinear ecuaciones dentro del modo matemático, de modo que podemos realizar construcciones algo más complejas.
\[
	\sigma = 
	\begin{aligned}
			a^2 &= b^2 + c^2 \\
			E &= m c^2
	\end{aligned}
\]

% Este modo es similar a aligned, pero nos introduce una llave en el lado izquierdo de donde lo pongamos.
\[
	f(x) =%
	\begin{cases}
		0 \quad &\text{si} \quad x \leq 0 \\
		1 \quad &\text{si} \quad x > 0
	\end{cases}
\]


% Subequations es un ambiente que nos sirve para numerar las ecuaciones con el mismo número y una letra distinta ( 21a, 21b, 21c...)
\begin{subequations}
	\begin{equation}
		a^2 = b^2 + c^2
	\end{equation}
	\begin{equation}
		E = m c^2	
	\end{equation}
	\begin{equation}
		\dif t = \xi m
	\end{equation}
\end{subequations}

\end{document}











