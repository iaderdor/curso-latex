\makeatletter
\def\input@path{{../}}
\makeatother

\documentclass[../preambulo.tex]{subfiles}


\begin{document}


% Comienza una sección. Es recomendable dejar un poco de espacio arriba y abajo para identificarlas más rápido.
\section{Inicio}
\label{Sec:inicio}	% Etiqueta que para hacer referencias a la sección.

Esto es una cita al artículo \cite{Ikemachi}.

Lorem ipsum dolor sit amet, consectetur adipiscing elit. Etiam vel diam ut felis posuere commodo eu ut dolor. Pellentesque eu accumsan ex, ac luctus massa. Nulla sit amet nulla auctor, condimentum mauris in, sollicitudin neque. Vestibulum ullamcorper id lacus non ullamcorper. Curabitur pharetra pretium mauris ut venenatis. Cras eget volutpat erat. Quisque interdum diam eu nibh porttitor finibus. Orci varius natoque penatibus et magnis dis parturient montes, nascetur ridiculus mus. Fusce sed pellentesque ipsum. Class aptent taciti sociosqu ad litora torquent per conubia nostra, per inceptos himenaeos. Sed sit amet metus ac leo ultricies mattis. Duis dictum sagittis nisl ac placerat. Ut elementum quis erat viverra dapibus.



\subsection{Teorema interesante}


Nunc vehicula aliquet risus, sed hendrerit leo suscipit eget. Phasellus ultrices mattis laoreet. Fusce porta lacus et massa fringilla, nec porta sem convallis. Ut finibus eu est in rhoncus. Morbi viverra massa id nibh porttitor, sed ullamcorper metus gravida. Integer at eleifend orci, vel lacinia lorem. Phasellus vitae nisi ipsum.


% Aquí definimos un elemento deslizante que queremos que sitúe exactamente aquí. El elemento contiene una tabla.
\begin{figure}[H]
	\centering
	\begin{tabular}{ccc}
	Nombre & Puntos & Tiempo \\ 
	Minkowski & 25pts & 3h 45m 
	\end{tabular}
\end{figure}


Sed commodo porttitor mauris. Cras venenatis enim sit amet sagittis volutpat. Morbi in interdum nunc, ac sagittis arcu. Mauris at mi gravida, sodales est eu, auctor sapien. Duis vulputate ac erat id pharetra. Maecenas ac interdum mi. Donec nulla sapien, mollis sed sapien eu, malesuada vulputate est. In vitae vulputate augue.





\end{document}